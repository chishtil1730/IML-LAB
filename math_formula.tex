\documentclass[12pt]{article.cls}
\usepackage{amsmath, amssymb}
\usepackage{geometry}
\geometry{margin=1in}

\title{Mean, Median, and Mode -- Formulae}
\author{}
\date{}

\begin{document}
\maketitle


\section*{1. Discrete (Ungrouped) Data}

Let the observations be
\[
x_1, x_2, \dots, x_n
\]

\subsection*{Mean}
\[
\bar{x} = \frac{\sum x}{n}
\]

\subsection*{Median}

Arrange the data in ascending order.

\begin{itemize}
    \item If \( n \) is odd:
    \[
    \text{Median} = \text{value of } \left(\frac{n+1}{2}\right)^{\text{th}} \text{ item}
    \]

    \item If \( n \) is even:
    \[
    \text{Median} =
    \frac{
    \left(\frac{n}{2}\right)^{\text{th}} \text{ item}
    +
    \left(\frac{n}{2}+1\right)^{\text{th}} \text{ item}
    }{2}
    \]
\end{itemize}

\subsection*{Mode}
\[
\text{Mode} = \text{value with the highest frequency}
\]

\section*{2. Discrete Grouped Data}

(Data given as values \( x_i \) with frequencies \( f_i \))

\subsection*{Mean}
\[
\bar{x} = \frac{\sum f_i x_i}{\sum f_i}
\]

\subsubsection*{Assumed Mean Method}
\[
\bar{x} = a + \frac{\sum f_i d_i}{\sum f_i}
\]

where
\[
d_i = x_i - a
\]
and \( a \) is the assumed mean.

\subsection*{Median}
\[
\text{Median} = \text{value whose cumulative frequency } \ge \frac{N}{2}
\]

where
\[
N = \sum f_i
\]

\subsection*{Mode}
\[
\text{Mode} = \text{value with maximum frequency}
\]

\section*{3. Continuous Grouped Data}

(Data in class intervals with frequencies)

\subsection*{Mean}
\[
\bar{x} = \frac{\sum f_i m_i}{\sum f_i}
\]

where
\[
m_i = \frac{\text{lower limit} + \text{upper limit}}{2}
\]

\subsection*{Step Deviation Method}
\[
\bar{x} = a + \frac{\sum f_i d_i}{\sum f_i} \times h
\]

where
\[
d_i = \frac{m_i - a}{h}
\]

\begin{itemize}
    \item \( h \) = class width
    \item \( a \) = assumed mean
\end{itemize}

\subsection*{Median}
\[
\text{Median} =
l + \left(\frac{\frac{N}{2} - cf}{f}\right) \times h
\]

where
\begin{itemize}
    \item \( l \) = lower boundary of median class
    \item \( N = \sum f \)
    \item \( cf \) = cumulative frequency before median class
    \item \( f \) = frequency of median class
    \item \( h \) = class width
\end{itemize}

\subsection*{Mode}
\[
\text{Mode} =
l + \left(\frac{f_1 - f_0}{2f_1 - f_0 - f_2}\right) \times h
\]

where
\begin{itemize}
    \item \( l \) = lower boundary of modal class
    \item \( f_1 \) = frequency of modal class
    \item \( f_0 \) = frequency of class before modal class
    \item \( f_2 \) = frequency of class after modal class
    \item \( h \) = class width
\end{itemize}

\section*{Empirical Relation (Optional)}
\[
\text{Mode} = 3(\text{Median}) - 2(\text{Mean})
\]

\section*{2.Discrete (Ungrouped) Data}

Let the observations be
\[
x_1, x_2, \dots, x_n
\]

\subsection*{Variance}
\[
\sigma^2 = \frac{1}{n} \sum (x - \bar{x})^2
\]

\subsection*{Standard Deviation}
\[
\sigma = \sqrt{\frac{1}{n} \sum (x - \bar{x})^2}
\]

\subsection*{Coefficient of Variation}
\[
\text{CV} = \frac{\sigma}{\bar{x}} \times 100
\]

\section*{Discrete Grouped Data}

(Data given as values \( x_i \) with frequencies \( f_i \))

\subsection*{Variance}
\[
\sigma^2 = \frac{\sum f_i (x_i - \bar{x})^2}{\sum f_i}
\]

\subsection*{Standard Deviation}
\[
\sigma = \sqrt{\frac{\sum f_i (x_i - \bar{x})^2}{\sum f_i}}
\]

\subsection*{Assumed Mean Method}
\[
\sigma = \sqrt{\frac{\sum f_i d_i^2}{\sum f_i} -
\left(\frac{\sum f_i d_i}{\sum f_i}\right)^2}
\]

where
\[
d_i = x_i - a
\]
and \( a \) is the assumed mean.

\subsection*{Coefficient of Variation}
\[
\text{CV} = \frac{\sigma}{\bar{x}} \times 100
\]
\section*{Continuous Grouped Data}

(Data given in class intervals with frequencies)

\subsection*{Variance}
\[
\sigma^2 = \frac{\sum f_i (m_i - \bar{x})^2}{\sum f_i}
\]

\subsection*{Standard Deviation}
\[
\sigma = \sqrt{\frac{\sum f_i (m_i - \bar{x})^2}{\sum f_i}}
\]

where
\[
m_i = \frac{\text{lower limit} + \text{upper limit}}{2}
\]

\subsection*{Step Deviation Method}
\[
\sigma = h \sqrt{
\frac{\sum f_i d_i^2}{\sum f_i}
-
\left(\frac{\sum f_i d_i}{\sum f_i}\right)^2
}
\]

where
\[
d_i = \frac{m_i - a}{h}
\]

\begin{itemize}
    \item \( h \) = class width
    \item \( a \) = assumed mean
\end{itemize}

\subsection*{Coefficient of Variation}
\[
\text{CV} = \frac{\sigma}{\bar{x}} \times 100
\]


\section*{Bayes' Theorem}

If \( A \) and \( B \) are events with \( P(B) \neq 0 \), then
\[
P(A \mid B) = \frac{P(B \mid A)\, P(A)}{P(B)}
\]

If \( A_1, A_2, \dots, A_n \) are mutually exclusive and exhaustive events, then
\[
P(A_i \mid B) =
\frac{P(B \mid A_i)\, P(A_i)}
{\sum_{j=1}^{n} P(B \mid A_j)\, P(A_j)}
\]

\section*{Multiplication Theorem of Probability}

For two events \( A \) and \( B \) with \( P(B) \neq 0 \),
\[
P(A \cap B) = P(A)\, P(B \mid A)
\]

Similarly,
\[
P(A \cap B) = P(B)\, P(A \mid B)
\]

\subsection*{For Three Events}
\[
P(A \cap B \cap C)
= P(A)\, P(B \mid A)\, P(C \mid A \cap B)
\]

If \( A, B, C \) are independent events, then
\[
P(A \cap B \cap C) = P(A)\, P(B)\, P(C)
\]

---

\section*{Inclusion--Exclusion Principle}

\subsection*{For Two Events}
\[
P(A \cup B) = P(A) + P(B) - P(A \cap B)
\]

\subsection*{For Three Events}
\[
\begin{aligned}
P(A \cup B \cup C) ={} & P(A) + P(B) + P(C) \\
& - P(A \cap B) - P(B \cap C) - P(C \cap A) \\
& + P(A \cap B \cap C)
\end{aligned}
\]

\subsection*{General Form (n Events)}
\[
P\left(\bigcup_{i=1}^{n} A_i\right)
=
\sum P(A_i)
-
\sum P(A_i \cap A_j)
+
\sum P(A_i \cap A_j \cap A_k)
-
\cdots
+ (-1)^{n-1} P(A_1 \cap \cdots \cap A_n)
\]


\section*{Probability Mass Function (PMF)}

Let \( X \) be a discrete random variable.

The probability mass function is defined as
\[
P(X = x) = p(x)
\]

Properties:
\begin{itemize}
    \item \( p(x) \ge 0 \) for all \( x \)
    \item \( \sum p(x) = 1 \)
\end{itemize}

\section*{Cumulative Distribution Function (CDF)}

The cumulative distribution function of a random variable \( X \) is
\[
F(x) = P(X \le x)
\]

\subsection*{For Discrete Random Variables}
\[
F(x) = \sum_{t \le x} P(X = t)
\]

\subsection*{For Continuous Random Variables}
\[
F(x) = \int_{-\infty}^{x} f(t)\, dt
\]

Properties:
\begin{itemize}
    \item \( 0 \le F(x) \le 1 \)
    \item \( F(-\infty) = 0 \)
    \item \( F(\infty) = 1 \)
\end{itemize}

\section*{Relation Between PDF and CDF}

If \( F(x) \) is differentiable, then
\[
f(x) = \frac{d}{dx} F(x)
\]


\end{document}
